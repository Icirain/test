Basically , at every this robot game initiated, one robot, three bases with fixed size and 4 obstacles with varied size will be placed in an area surrounded by white line, which is the boundaries of this area. Basically this game are built with a standard M\+VC style which contains model, view and controller. Viewer is the part of code which has a update function repeatedly called behind to draw the whole graphic part of this game. \hyperlink{classArena}{Arena} is the backend model which records the status like velocity, position and if\+\_\+captured for base. All of these data is stored in an instance of arena class implemented in \hyperlink{arena_8h}{arena.\+h} \& \hyperlink{arena_8cc}{arena.\+cc}. The instances of robot, bases and obstacles are the composition of arena so that arena has access to all of their data and able to modify them if necessary.

To implement the communication between viewer and arena(model), functionality of controller is implemented mainly in \hyperlink{controller_8cc}{controller.\+cc} and \hyperlink{controller_8cc}{controller.\+cc}. When some activity in the viewer model activated like buttion pressed, if will be sent to controller immediately and finally forwarded to arena finally. In our code the instance of viewer and arena is the private variable in controller, which enables controller to send communication code to arena with function “\+Accept\+Communication()” inside arena. For example when button in the viewer called the relative communication code will be sent to controller and controller will forward it to arena to produce necessary response, which is the base stone of our game’s interaction mechanism.

To control the robot’s velocity and direction, we use arrow key to do that. When any arrow key pressed , the viewer’ call back function will be called and relative communication, which is listed in \hyperlink{communication_8h}{communication.\+h}, will be sent to controller and finally to the arena. For instance when up arrow key pressed, the velocity of robot inside arena will be increased and vice versa. Arrow key of left and right is to control the direction of robot which will lead to different velocity on two sides of robot. With same logic , the functionality of new game, pause\& play are implemented. When button, corresponding communication will be forwarded to arena through controller.

For the play\&pause funtionality, when button pressed. A variable of arena will be modified between true and false whose name is is\+\_\+playing. If is\+\_\+playing is true the regulated time interaction named dt will be passed to Advance\+T\+Ime() function in arena which drives the update of all the instances in arena. But if pause button pressed in viewer, Advance\+Time function will stopped ahead of the update happening to implement the pause of whole game. Graphically all stuff in the playing area will keep not moving in paused status. And for new game functionality. When this button named “\+New Game” pressed, in the \hyperlink{graphics__arena__viewer_8cc}{graphics\+\_\+arena\+\_\+viewer.\+cc}, the old arena instance inside viewer will be deleted and a new one will be created. And the communication code “k\+New\+Game” will be sent to controller and the new arena instance will be redirected inside controller to prevent segment fault.

In fixed time period dt , the game is proceeding and the viewer will be updated in every dt. And collision is the major functionality implemented in this iteration. For two kinds of mobile entity obstacle and robot. Whenever the robot collision happens, in the funtion Handle\+Collision, the lives will be deducted and speed will be set to 0. For obstacle, it will reverse direction and finish an arc to avoid double collison. For the whole game rule, the player needs to control the speed and direction of robot to capture all 3 bases before losing all lives. Whenver the lives is 0 or all bases captured the game will end up. Player needs to press new game button to restart a new game. And in the control area the status of game is descripted and updated.

Have fun with this game!\hypertarget{index_intro_sec}{}\section{Introduction}\label{index_intro_sec}
This is the introduction. 